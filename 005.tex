\documentclass[11pt,letterpaper]{article}
\usepackage[utf8]{inputenc}
\usepackage{amsmath}
\usepackage{amsfonts}
\usepackage{amssymb}
\usepackage{makeidx}
\usepackage{graphicx}
\usepackage[left=2.5cm,right=2cm,top=2cm,bottom=2cm]{geometry}
\author{Héctor Jesús Ponce Castillo}
\title{Plantilla de LaTeX}
\begin{document}
%Ahora como generamos el indice, de un documento en latex es asi de sencillo
\tableofcontents
\newpage
\section{Introducción}
	\subsection{Jajaja}
\newpage
\section{Resumen}
	\subsection{Jajaja}
\newpage
\section{Planteamiento del Problema}
	\subsection{Jajaja}
\newpage
\section{Parte 4}
	\subsection{Jajaja}
\newpage
\section{Parte 5}
	\subsection{Jajaja}
\newpage
\section{Parte 6}
	\subsection{Jajaja}
\end{document}